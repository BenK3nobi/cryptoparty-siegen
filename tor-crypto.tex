\documentclass{beamer}
\usepackage{graphicx}
\usepackage[ngerman]{babel}
\usepackage[utf8]{inputenc}
\usetheme{Warsaw}  %% Themenwahl


% neuer Befehl: \includegraphicstotab[..]{..}
% Verwendung analog wie \includegraphics
\newlength{\myx} % Variable zum Speichern der Bildbreite
\newlength{\myy} % Variable zum Speichern der Bildhöhe
\newcommand\includegraphicstotab[2][\relax]{%
% Abspeichern der Bildabmessungen
\settowidth{\myx}{\includegraphics[{#1}]{#2}}%
\settoheight{\myy}{\includegraphics[{#1}]{#2}}%
% das eigentliche Einfügen
\parbox[c][1.1\myy][c]{\myx}{%
\includegraphics[{#1}]{#2}}%
}% Ende neuer Befehl


\title{Anonym surfen mit Tor}
\author{Tobias Becker}
\date{\today}

\begin{document}
\maketitle
%\frame{\tableofcontents[currentsection]}

\section{Tor Funktion}
\begin{frame}
\begin{itemize}
\item Erste Idee 2000 an der Universität Cambridge
\item 2002 Erste Alpha Version
\item 2001-2006 wurde die Entwicklung unterstützt von United States Naval Research Laboratory (NRL), Office of Naval Research (ONR) und Defense Advanced Research Projects Agency (DARPA)
\item 2004-2005 Unterstützung der EFF
\item 2012 60\% Finanzierung durch US-Regierung
\end{itemize}
\end{frame}
\begin{frame}
\includegraphicstotab[width=1.0\linewidth]{Bilder/TOR.png}
\end{frame}

\section{Tor Installation}
\begin{frame}{How to get TOR?}
\begin{block}{Tor-Browser-Bundle}
https://www.torproject.org/download/download-easy.html.en
\end{block}
\end{frame}
\begin{frame}{How to navigate?}
\begin{block}{Hidden Wiki}
http://zqktlwi4fecvo6ri.onion/wiki/index.php/Main\_Page
\end{block}
\end{frame}
\end{document}
